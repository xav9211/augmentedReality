\documentclass[pdflatex,11pt]{aghdpl}
% \documentclass{aghdpl}               % przy kompilacji programem latex
% \documentclass[pdflatex,en]{aghdpl}  % praca w języku angielskim
\usepackage[polish]{babel}
\usepackage[utf8]{inputenc}

% dodatkowe pakiety
\usepackage{enumerate}
\usepackage{listings}
\usepackage{graphicx}
\lstloadlanguages{TeX}

\lstset{
  literate={ą}{{\k{a}}}1
           {ć}{{\'c}}1
           {ę}{{\k{e}}}1
           {ó}{{\'o}}1
           {ń}{{\'n}}1
           {ł}{{\l{}}}1
           {ś}{{\'s}}1
           {ź}{{\'z}}1
           {ż}{{\.z}}1
           {Ą}{{\k{A}}}1
           {Ć}{{\'C}}1
           {Ę}{{\k{E}}}1
           {Ó}{{\'O}}1
           {Ń}{{\'N}}1
           {Ł}{{\L{}}}1
           {Ś}{{\'S}}1
           {Ź}{{\'Z}}1
           {Ż}{{\.Z}}1
}

%---------------------------------------------------------------------------

\author{Ksawery Głaz}

\titlePL{Analiza wpływu składowych systemu rozszerzonej rzeczywistości na jakość uzyskiwanego obrazu.}
% \titleEN{}

% \shorttitlePL{Przygotowanie pracy dyplomowej w~systemie \LaTeX} % skrócona wersja tytułu jeśli jest bardzo długi
% \shorttitleEN{Thesis in \LaTeX}

\thesistypePL{Praca magisterska}
% \thesistypeEN{Master of Science Thesis}

\supervisorPL{prof. dr hab. inż. Krzysztof Boryczko}
% \supervisorEN{Marcin Szpyrka Ph.D}

\date{2016}

\departmentPL{Katedra Informatyki}
% \departmentEN{Department of Automatics}

\facultyPL{Wydział Informatyki, Elektroniki i Telekomunikacji}
% \facultyEN{Faculty of Electrical Engineering, Automatics, Computer Science and Electronics}

% \acknowledgements{Serdecznie dziękuję \dots tu ciąg dalszych podziękowań np. dla promotora, żony, sąsiada itp.}


\setlength{\cftsecnumwidth}{10mm}

%---------------------------------------------------------------------------

\begin{document}

\titlepages

\tableofcontents
\clearpage

\chapter{Wprowadzenie}
\label{cha:wprowadzenie}

\par
Modelowanie w ostatnich latach stało się istotnym zagadnieniem w wielu dziedzinach naukowych. Podstawowym celem modelowania stało się uproszczenie rzeczywistości umożliwiające poddanie jej procesowi badawczemu. Takie podejście umożliwia między innymi analizę zjawiska będącego przedmiotem badań w zmienionej skali czasowo-przestrzennej, adekwatną zmianę skali obiektu badań, analizę procesów zachodzących w skalach czasowych rzędu nano- lub mikrosekund czyli trudnych do uchwycenia w warunkach laboratoryjnych. Modelowanie umożliwia również badanie wybranego, jednego aspektu zagadnienia będącego przedmiotem analizy.
\par
Odrębnym zagadnieniem w technikach modelowania jest modelowanie obiektów i scen trójwymiarowych. Dotyczy ono tworzenia obiektów i scen wykorzystujących głównie obiekty dostępne w bibliotekach różnych programów i zasadniczo ogranicza się do scen statycznych. Znacznie bardziej ciekawym rozwiązaniem jest propozycja połączenia scen pochodzących ze świata rzeczywistego z obiektami lub scenami generowanymi komputerowo. Klasyczne w tym zakresie rozwiązania bazują na połączeniu obrazu świata rzeczywistego dostarczanego przez kamerę z generowaną w czasie rzeczywistym grafiką 3D. Należy zwrócić uwagę, iż obraz świata rzeczywistego może być dostarczany przez sygnały o różnych częstotliwościach podstawowych. Spotyka się również rozwiązania, w których wykorzystywana jest fuzja sygnałów z kamer pracujących w zakresie fal widzialnych dla człowieka oraz fal radiowych. Powstały w ten sposób obraz określa się rozszerzoną rzeczywistością (ang. Augmented Reality).
\par
W literaturze można spotkać wiele definicji rozszerzonej rzeczywistości. Najistotniejsze wydają się być jednak wymagania stawiane tego typu systemom. Wśród najważniejszych należy wymienić przede wszystkim łączenie świata rzeczywistego z wirtualnym, konieczność pracy w czasie rzeczywistym oraz umożliwianie ruchów każdego elementu w trzech wymiarach.
\par
Pozornie prosty postulat dotyczący łączenia scen świata rzeczywistego z generowanymi scenami wirtualnymi kryje w sobie konieczność rozwiązania szeregu problemów naukowych oraz technologicznych. Efektywność zaproponowanych w tym zakresie metod przekłada się bezpośrednio na jakość uzyskanej sceny (modelowania). Dodatkowe ograniczenie wymuszające konieczność generowania wynikowej sceny w czasie rzeczywistym wymusza użycie właściwych dla danego zastosowania algorytmów dedykowanych dla danej architektury sprzętowej oraz ich specyficzną implementację.

%---------------------------------------------------------------------------

\section{Cele pracy}
\label{sec:celePracy}

Celem głównym niniejszej pracy jest stworzenie systemu do tworzenia aplikacji z zakresu rozszerzonej rzeczywistości. Szczególny nacisk zostanie położony na jakość uzyskiwanych scen, co jest zdeterminowane głównie precyzją łączenia obrazu rzeczywistego i scen wirtualnych. Dla tak zaproponowanego celu ogólnego zrealizowano kilka celów szczegółowych. Należą do nich:
\begin{itemize}
	\item Analiza dostępnych algorytmów łączenia obrazu rzeczywistego i scen generowanych.
	\item Analiza metod oceny jakości sceny rozszerzonej rzeczywistości oraz propozycja własnych kryteriów w tym zakresie.
	\item Propozycja własnych algorytmów dla metod rozszerzonej rzeczywistości uwzględniających określone kryteria jakości.
	\item Ocenę jakości zaproponowanych algorytmów.
	\item Sformułowanie ogólnych zasad tworzenia aplikacji dla przedmiotowego zakresu.
\end{itemize}

%---------------------------------------------------------------------------

\section{Zawartość pracy}
\label{sec:zawartoscPracy}

Niniejsza praca zawiera opis realizacji merytorycznie spójnych etapów koniecznych dla stawianego celu ogólnego. W rozdziale pierwszym ....



\chapter{Przegląd istniejących rozwiązań w zakresie rozszerzonej rzeczywistości}
\label{cha:przegladIstniejacychRozwiazanWZakresieRozszerzonejRzeczywistosci}

W tym rozdziale zostało omówione kilka bibliotek rozwiązujących zagadnienie rozszerzonej rzeczywistości.

%---------------------------------------------------------------------------

% \section{Struktura dokumentu}
% \label{sec:strukturaDokumentu}


% %---------------------------------------------------------------------------

% \section{Kompilacja}
% \label{sec:kompilacja}



% %---------------------------------------------------------------------------

% \section{Narzędzia}
% \label{sec:narzedzia}



% %---------------------------------------------------------------------------

% \section{Przygotowanie dokumentu}
% \label{sec:przygotowanieDokumentu}


\chapter{Propozycja autorskiego rozwiązania}
\label{cha:propozycjaAutorskiegoRozwiazania}

Zaproponowane autorskie rozwiązanie.



\chapter{Ocena jakości}
\label{cha:ocenaJakosci}

Ocena jakości.



\chapter{Podsumowanie i wnioski}
\label{cha:PodsumowanieIWnioski}

Podsumowanie i wnioski.






% itd.
% \appendix
% \include{dodatekA}
% \include{dodatekB}
% itd.

\bibliographystyle{alpha}
\bibliography{bibliografia}
\begin{thebibliography}{1}

\bibitem{ARToolkit}
Ben Vaughan, Philip Lamb, Wally Young.
\newblock {\em ARToolkit}.
\newblock \texttt{http://www.artoolworks.com/}
\newblock \testtt{https://github.com/artoolkit}

\bibitem{ArUco}
Rafael Munoz-Salinas.
\newblock {\em ArUco: a minimal library for Augmented Reality applications based on OpenCV}.
\newblock \texttt{http://www.uco.es/investiga/grupos/ava/node/26}

\bibitem{AugmentedRealityChess}
Augmented Reality Chess
\newblock \texttt{https://www.youtube.com/watch?v=sqAIvO6wCQI}

\bibitem{Vuforia}
\newblock {\em {Vufora Developer Portal}}.
\newblock \texttt{https://developer.vuforia.com}.

\end{thebibliography}

\end{document}
